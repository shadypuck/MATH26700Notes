\usepackage[margin=1in]{geometry}
\usepackage{csquotes}
\usepackage{fancyhdr}
\usepackage{marginnote}
\usepackage{enumitem}
\usepackage{scrextend}
\usepackage[bottom]{footmisc}
\usepackage{xr}
\usepackage[style=apa]{biblatex}
\usepackage{tikz}
\usepackage{float,subcaption}
\usepackage{amsmath,amssymb,amsbsy,amsthm}
\usepackage{bm,ytableau,nicematrix,physics,mathtools}
\usepackage[hidelinks]{hyperref}
\usepackage{subfiles}

\MakeOuterQuote{"}

\fancypagestyle{main}{
    \fancyhf{}
    \fancyhead[L]{\leftmark}
    \fancyhead[R]{MATH 26700}
    \fancyfoot[R]{Labalme\ \thepage}
}
\fancypagestyle{plain}{
    \fancyhead{}
    \renewcommand{\headrulewidth}{0pt}
}

\reversemarginpar

\setitemize[3]{label={\scriptsize$\blacksquare$}}

\deffootnotemark{\textsuperscript{\textup{[}\thefootnotemark\textup{]}}}
\deffootnote[1.8em]{0em}{0em}{\textsuperscript{\thefootnote}}

\addbibresource{\subfix{../main.bib}}
\DefineBibliographyStrings{english}{bibliography={References}}

\usetikzlibrary{calc,cd,bending}
\colorlet{blx}{blue!90!green!80}
\colorlet{bly}{blue!90!green!60}
\colorlet{blz}{blue!90!green!20}

\DeclareMathOperator{\sign}{sign}
\DeclareMathOperator{\Hom}{Hom}
\DeclareMathOperator{\im}{Im}
\DeclareMathOperator{\evl}{ev}
\DeclareMathOperator{\Ob}{Ob}
\DeclareMathOperator{\id}{id}
\DeclareMathOperator{\Rep}{Rep}
\DeclareMathOperator{\spn}{span}
\DeclareMathOperator{\coker}{Coker}
\let\ker\relax
\DeclareMathOperator{\ker}{Ker}
\DeclareMathOperator{\diag}{diag}
\DeclareMathOperator{\Fix}{Fix}

\newtheorem{theorem}{Theorem}
\newcounter{FHchapter}
\newtheorem{FHlemma}{Lemma}[FHchapter]
\newtheorem{FHproposition}[FHlemma]{Proposition}

\ytableausetup{smalltableaux}

\newcommand{\N}{\mathbb{N}}
\newcommand{\Z}{\mathbb{Z}}
\newcommand{\R}{\mathbb{R}}
\newcommand{\C}{\mathbb{C}}
\newcommand{\F}{\mathbb{F}}

\newcommand{\e}[1][]{\text{e}^{#1}}
\newcommand{\inp}[1]{\left\langle{#1}\right\rangle}

\DeclareFontFamily{U}{mathb}{}
\DeclareFontShape{U}{mathb}{m}{n}{
       <-5.5> mathb5
    <5.5-6.5> mathb6
    <6.5-7.5> mathb7
    <7.5-8.5> mathb8
    <8.5-9.5> mathb9
    <9.5-11>  mathb10
    <11->     mathb12
}{}
\DeclareSymbolFont{mathb}{U}{mathb}{m}{n}
\DeclareMathSymbol{\righttoleftarrow}{3}{mathb}{"FD}
% 
\makeatletter
\DeclareRobustCommand{\looparrow}[1]{%
  \mathrel{\mathpalette\looparrow@{#1}}%
}
\newcommand{\looparrow@}[2]{\reflectbox{$\m@th#1#2$}}
\makeatother
% 
\newcommand{\acts}{\looparrow{\righttoleftarrow}}