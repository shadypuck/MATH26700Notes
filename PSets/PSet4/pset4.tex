\documentclass[../psets.tex]{subfiles}

\pagestyle{main}
\renewcommand{\leftmark}{Problem Set \thesection}
\setcounter{section}{3}

\begin{document}




\section{More Characters and Intro to Associative Algebras}
\begin{enumerate}
    \item \marginnote{10/27:}\textbf{Representations of $\bm{S_5}$}.
    \begin{enumerate}
        \item Prove that there exist only two one-dimensional representations of $S_5$: The trivial representation $(5)$ and the alternating representation $(1,1,1,1,1)$.
        \item Compute the character of the standard representation $(4,1)$ by decomposing the permutational representation into irreducibles.
        \item Prove that the representations $(3,1,1)=\Lambda^2(4,1)$ and $(2,1,1,1)=\Lambda^3(4,1)$ are irreducible. Compute their characters. Prove that $(2,1,1,1)=(1,1,1,1,1)\otimes(4,1)$.
        \item Find the two remaining irreducible representations of $S_5$; denote them $(3,2)$ and $(2,2,1)$. Complete the character table.
        \item Consider an exceptional homomorphism $S_5\to S_6$. Decompose the corresponding permutational representation into irreducibles.
        \item The \textbf{Petersen graph} is a graph with vertices being 2-element subsets of $\{1,2,3,4,5\}$; two vertices are connected by an edge if the corresponding sets do not intersect (see \href{https://en.wikipedia.org/wiki/Petersen_graph}{wiki} for a picture).
        \begin{center}
            \includegraphics[width=0.3\linewidth]{PetersenGraph.png}
        \end{center}
        Consider a natural action of $S_5$ on the set of complex-valued functions on the set of vertices of the Peterson graph. Find the character of the corresponding representation $V$ and decompose it into irreducibles.
        \item Decompose $V$ into isotypical components.
        \item Consider an endomorphism $A:V\to V$ sending a function to the average of its values on the adjacent vertices. Prove that it is an endomorphism of the corresponding representation. Find the spectrum of $A$.
    \end{enumerate}
    \item The character table is a square matrix. Determine the absolute value of its determinant.
    \item Prove that for any irreducible character $\chi$ of a group $G$, we have
    \begin{equation*}
        \chi(g)\chi(h) = \frac{d_\chi}{|G|}\sum_{x\in G}\chi(gxhx^{-1})
    \end{equation*}
    \item Let $F$ be a field.
    \begin{enumerate}
        \item Prove that the matrix algebra $M_{n,n}(F)$ is simple, i.e., has no nontrivial ideals.
        \begin{proof}
            To prove that $M_{n,n}(F)$ is simple, it will suffice to show that any nonzero ideal of it contains the identity. Let $I$ be a nonzero ideal of $M_{n,n}(F)$. Then there exists some nonzero $A\in I$. In particular, if $A$ is nonzero, then it contains some entry $a_{ij}\neq 0$. Now, let $E_{ij}\in M_{n,n}(F)$ be the matrix with 1 in the $i,j$ position and 0 everywhere else. Then for each $k=1,\dots,n$,
            \begin{equation*}
                a_{ij}^{-1}E_{ki}AE_{jk} = E_{kk}
            \end{equation*}
            so $E_{kk}\in I$. Summing all of the $E_{kk}$'s within $I$ yields the $n\times n$ identity matrix, as desired.
        \end{proof}
        \item Prove that there is a unique simple module over the matrix algebra $M_{n,n}(F)$.
        \begin{proof}
            % \begin{itemize}
            %     \item $M_{n,n}(F)$ is a semisimple algebra.
            %     \begin{itemize}
            %         \item WTS: $M_{n,n}(F)\cong(F^n)^n$ as modules.
            %         \item Define $T:M_{n,n}(F)\to(F^n)^n$ by $(v_1\mid\cdots\mid v_n)\mapsto v_1\oplus\cdots\oplus v_n$.
            %         \item WTS: $T$ is a $F$-module homomorphism (i.e., a linear map).
            %         \item It is fairly obvious that $T(f_1A_1+f_2A_2)=f_1T(A_1)+f_2T(A_2)$ (i.e., $T$ is a homomorphism of abelian groups and commutes with scalar multiplication), so we're all good to go.
            %     \end{itemize}
            %     \item Let $S$ be a simple $M_{n,n}(F)$-module.
            %     \item $M_{n,n}(F)$ is a semisimple algebra: $S$ is semisimple module.
            %     \item Monday's theorem: $S=\bigoplus_{i\in I}S_i=(F^n)^N\oplus S_1^{n_1}\oplus\cdots$.
            %     \item Wikipedia: $S=M_{n,n}(F)s$ for any $s\in S$.
            %     \item $F^n=M_{n,n}(F)e_1$.
            %     \item Define $f:S\to F^n$ by $As\mapsto Ae_1$.
            %     \item $f$ is a module homomorphism.
            %     \begin{itemize}
            %         \item $f$ is a homomorphism of abelian groups.
            %         \begin{equation*}
            %             f(As+Bs) = f[(A+B)s]
            %             = (A+B)e_1
            %             = Ae_1+Be_1
            %             = f(As)+f(Bs)
            %         \end{equation*}
            %         \item $f$ commutes with scalar multiplication.
            %         \begin{equation*}
            %             f(B(As)) = f[(BA)s]
            %             = (BA)e_1
            %             = B(Ae_1)
            %             = Bf(As)
            %         \end{equation*}
            %     \end{itemize}
            %     \item $f$ is injective.
            %     \begin{itemize}
            %         \item $\ker(f)$ is a submodule of $S$.
            %         \item $S$ is simple: $\ker(f)=0$.
            %     \end{itemize}
            %     \item $f$ is surjective.
            %     \begin{itemize}
            %         \item $\im(f)$ is a submodule of $F^n$.
            %         \item $F^n$ is simple: $\im(f)=F^n$.
            %     \end{itemize}
            %     \item $f$ is an injective, surjective, module homomorphism: $f$ is a module isomorphism.
            %     \item Thus, $S\cong F^n$.
            % \end{itemize}

            We will prove that $F^n$ is a simple module over $M_{n,n}(F)$, and then that every simple module over $M_{n,n}(F)$ is isomorphic to $F^n$. Note that the action of $M_{n,n}(F)$ on $F^n$ is defined to be left matrix multiplication. Let's begin.\par\smallskip
            To prove that $F^n$ is a simple module, it will suffice to show that any nonzero submodule of $F^n$ equals $F^n$. Let $N$ be an arbitrary nonzero submodule of $F^n$. To show that $N=F^n$, we will use a bidirectional inclusion proof. As a submodule of $F^n$, we immediately have $N\subset F^n$. In the other direction, let $y=(y_1,\dots,y_n)\in F^n$ be arbitrary. Since $N$ is nonzero, it contains some nonzero element $x=(x_1,\dots,x_n)$. Suppose $x_j\neq 0_F$. Let $A\in M_{n,n}(F)$ be zero everywhere except in column $j$, where we choose $a_{ij}=y_ix_j^{-1}$ ($i=1,\dots,n$). Thus, $y=Ax$, so $y\in N$, as desired.\par\smallskip
            Now let $S$ be an arbitrary simple module over $M_{n,n}(F)$. Let $s\in S$ nonzero be arbitrary. To prove that $S\cong F^n$ as a module, we will show that $S=M_{n,n}(F)s$, use this definition of $S$ to construct a map $f:S\to F^n$, and verify that this map is a module isomorphism. Let's begin.\par
            Consider the set
            \begin{equation*}
                M_{n,n}(F)s = \{As\mid A\in M_{n,n}(F)\}
            \end{equation*}
            Since $0\neq s\in M_{n,n}(F)s$, $As+Bs=(A+B)s\in M_{n,n}(F)s$, and $(-A)s\in M_{n,n}(F)s$ for all $As\in M_{n,n}(F)s$, we know that $M_{n,n}(F)s\leq S$. Since, additionally, $B(As)=(BA)s\in M_{n,n}(F)s$ for all $B\in M_{n,n}(F)$ and $As\in M_{n,n}(F)s$, we know that $M_{n,n}(F)s$ is a (nonzero) submodule of $S$. Consequently, since $S$ is simple, $M_{n,n}(F)s=S$. As a special case, note that $F^n$ being a simple $M_{n,n}(F)$-module implies that $F^n=M_{n,n}(F)e_1$, for instance.\par
            Define $f:S\to F^n$ by
            \begin{equation*}
                As \mapsto Ae_1
            \end{equation*}
            We have that $f$ is a homomorphism of abelian groups because
            \begin{equation*}
                f(As+Bs) = f[(A+B)s]
                = (A+B)e_1
                = Ae_1+Be_1
                = f(As)+f(Bs)
            \end{equation*}
            We have that $f$ commutes with scalar multiplication because
            \begin{equation*}
                f(B(As)) = f[(BA)s]
                = (BA)e_1
                = B(Ae_1)
                = Bf(As)
            \end{equation*}
            Thus, $f$ is a module homomorphism. Additionally, since $\ker(f)$ is a submodule of $S$, $S$ is simple, and $f$ is nontrivial, we have that $\ker(f)=0$ and hence $f$ is injective. Similarly, since $\im(f)$ is a submodule of $F^n$, $F^n$ is simple, and $f$ is nontrivial, we have that $\im(f)=F^n$ and hence $f$ is surjective. Thus, as an injective, surjective module homomorphism, $f$ is a module \emph{isomorphism} as well. Therefore, $S\cong F^n$, as desired.
        \end{proof}
    \end{enumerate}
    \item Consider the quaternion group $Q_8=\{\pm 1,\pm i,\pm j,\pm k\}\subset\HH$ of order 8.
    \begin{enumerate}
        \item Find four 1-dimensional representations of $Q_8$. Find the character of the remaining 2-dimensional representation.
        \begin{proof}
            Take the trivial representation.\par
            Then take the representations sending $i,j$ to $(-1)$, $i,k$ to $(-1)$, and $j,k$ to $(-1)$; everything else gets sent to the identity in each case.\par
            Working out the character table at this point, we can see that the character for the final representation must be given explicitly by the following.
            \begin{align*}
                +1 &\mapsto 2&
                -1 &\mapsto -2&
                +i &\mapsto 0&
                -i &\mapsto 0&
                +j &\mapsto 0&
                -j &\mapsto 0&
                +k &\mapsto 0&
                -k &\mapsto 0
            \end{align*}
        \end{proof}
        \item Prove that $\R[Q_8]\cong\R\oplus\R\oplus\R\oplus\R\oplus\HH$ as an algebra.
    \end{enumerate}
\end{enumerate}




\end{document}