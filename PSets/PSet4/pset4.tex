\documentclass[../psets.tex]{subfiles}

\pagestyle{main}
\renewcommand{\leftmark}{Problem Set \thesection}
\setcounter{section}{3}

\begin{document}




\section{Advanced Character Theory and Introduction to Associative Algebras}
\begin{enumerate}
    \setcounter{enumi}{3}
    \item \marginnote{10/27:}Let $F$ be a field.
    \begin{enumerate}
        \item Prove that the matrix algebra $M_{n,n}(F)$ is simple, i.e., has no nontrivial ideals.
        \begin{proof}
            To prove that $M_{n,n}(F)$ is simple, it will suffice to show that any nonzero ideal of it contains the identity. Let $I$ be a nonzero ideal of $M_{n,n}(F)$. Then there exists some nonzero $A\in I$. In particular, if $A$ is nonzero, then it contains some entry $a_{ij}\neq 0$. Now, let $E_{ij}\in M_{n,n}(F)$ be the matrix with 1 in the $i,j$ position and 0 everywhere else. Then for each $k=1,\dots,n$,
            \begin{equation*}
                \frac{1}{a_{ij}}E_{ki}AE_{jk} = E_{kk}
            \end{equation*}
            so $E_{kk}\in I$. Summing all of the $E_{kk}$'s within $I$ yields the $n\times n$ identity matrix, as desired.
        \end{proof}
        % \item Prove that there is a unique simple module over the matrix algebra $M_{n,n}(F)$.
    \end{enumerate}
    \item Consider the quaternion group $Q_8=\{\pm 1,\pm i,\pm j,\pm k\}\subset\HH$ of order 8.
    \begin{enumerate}
        \item Find four 1-dimensional representations of $Q_8$. Find the character of the remaining 2-dimensional representation.
        \begin{proof}
            Take the trivial representation.\par
            Then take the representations sending $i,j$ to $(-1)$, $i,k$ to $(-1)$, and $j,k$ to $(-1)$; everything else gets sent to the identity in each case.\par
            Working out the character table at this point, we can see that the character for the final representation must be given explicitly by the following.
            \begin{align*}
                +1 &\mapsto 2&
                -1 &\mapsto -2&
                +i &\mapsto 0&
                -i &\mapsto 0&
                +j &\mapsto 0&
                -j &\mapsto 0&
                +k &\mapsto 0&
                -k &\mapsto 0
            \end{align*}
        \end{proof}
    \end{enumerate}
\end{enumerate}




\end{document}