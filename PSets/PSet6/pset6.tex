\documentclass[../psets.tex]{subfiles}

\pagestyle{main}
\renewcommand{\leftmark}{Problem Set \thesection}
\setcounter{section}{5}

\begin{document}




\section{The Symmetric Group and Polynomials}
\begin{enumerate}
    \item \marginnote{11/17:}Let $G$ be a group of symmetries of a cube.
    \begin{enumerate}
        \item Prove that $|G|=48$ and that $G$ has a normal subgroup isomorphic to $S_4$.
        \item Prove that $G$ is isomorphic to a group of signed permutations, i.e., linear maps
        \begin{equation*}
            (x_1,x_2,x_3) \mapsto (\pm x_{\sigma(1)},\pm x_{\sigma(2)},\pm x_{\sigma(3)})
        \end{equation*}
        for $\sigma\in S_3$.
        \item Find the character table of $G$.
    \end{enumerate}
    \item For a partition $\lambda=(\lambda_1,\dots,\lambda_k)$ of $n$, define the conjugate partition $\lambda'=(\lambda_1',\dots,\lambda_{k'}')$ by formula $\lambda_i'=\{\#j\mid\lambda_j\geq i\}$. Prove that $\lambda'$ is also a partition of $n$ and $(\lambda')'=\lambda$ \emph{without} using the geometric picture.
    \item Let $G$ be a finite group, $Z(G)$ its center, and $V$ an irreducible representation of it. We proved in class that $\dim(V)$ divides the order of $G$. Prove the stronger statement that $\dim(V)$ divides the index $(G:Z(G))$.
    \item Prove the uniqueness part of the fundamental theorem about symmetric polynomials.
    \item Let $G$ be a finite non-abelian simple group. Every 1-dimensional representation of $G$ is trivial (why?).
    Prove that any 2-dimensional representation of $G$ is trivial as follows. Suppose that $\rho:G\to GL_2(\C)$ is nontrivial.
    \begin{enumerate}
        \item Prove that $G$ has an element $x$ of order 2.
        \item Prove that $\rho(x)=-\id$.
        \item Prove that $\rho([g,x])=\id$ for any $g$ and deduce the theorem.
    \end{enumerate}
    \item Consider the ring of symmetric polynomials $R=\Q[x_1,\dots,x_n]^{S_n}$.
    \begin{enumerate}
        \item Define the \textbf{power-sum symmetric polynomials} as follows.
        \begin{equation*}
            p_k(x_1,\dots,x_n) := x_1^k+\cdots+x_n^k
        \end{equation*}
        Prove the Newton formulas
        \begin{equation*}
            me_m(x_1,\dots,x_n) = \sum_{i=1}^m(-1)^{i-1}e_{m-i}(x_1,\dots,x_n)p_i(x_1,\dots,x_n)
        \end{equation*}
        Prove that $R=\Q[p_1,\dots,p_n]$.
        \item Define the \textbf{complete symmetric polynomials} $h_k(x_1,\dots,x_n)$ as a sum of all \emph{distinct} monomials of degree $k$. For instance,
        \begin{equation*}
            h_3(x_1,x_2) = x_1^3+x_2^3+x_1^2x_2+x_1x_2^2+x_1x_2x_3
        \end{equation*}
        Prove that
        \begin{equation*}
            \sum_{i=0}^m(-1)^ie_i(x_1,\dots,x_n)h_{m-i}(x_1,\dots,x_n) = 0
        \end{equation*}
        Prove that $R=\Q[h_1,\dots,h_n]$.
    \end{enumerate}
    \item Compute explicitly the characters of all representations $V_\lambda$ of $S_4$ using the construction with Specht polynomials. Check that you get the same results as we have obtained before.
\end{enumerate}




\end{document}