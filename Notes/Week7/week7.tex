\documentclass[../notes.tex]{subfiles}

\pagestyle{main}
\renewcommand{\chaptermark}[1]{\markboth{\chaptername\ \thechapter\ (#1)}{}}
\setcounter{chapter}{6}

\begin{document}




\chapter{Representations of the Symmetric Group}
\section{Specht Modules}
\begin{itemize}
    \item \marginnote{11/6:}Announcements.
    \begin{itemize}
        \item Midterm description is on the Canvas page.
        \item Review what he says to review, and then look at the PSets. The operator averaging stuff and $S_4$, $S_5$ examples are most important.
        \item New HW will be due next Friday (not this Friday).
    \end{itemize}
    \item New topic: Representations of $S_n$.
    \begin{itemize}
        \item We will talk about these almost until the end of the course.
        \item Very hard.
        \item Any specialist in rep theory will still say that they know some approaches, but nobody understands this stuff completely.
        \item We'll explore some phenomena, but if we feel after this course that we still don't understand everything about $S_n$, that's typical; if we think we understand everything, we're probably wrong.
    \end{itemize}
    \item Representation theory of $GL_n(\F_{p^k})$ is related but even worse.
    \begin{itemize}
        \item Same with $O_n(\F_{p^k})$.
        \item Recently, all this stuff was understood with something called linquistic (??) theory, but that's far beyond us.
    \end{itemize}
    \item $|S_n|=n!$, and the conjugacy classes are in bijection with cyclic structures of a permutation.
    \begin{itemize}
        \item Our good understanding of the conjugacy classes of $S_n$ is the only thing that makes this problem the slightest bit tractable.
        \item Cyclic structures are also in bijection with the \textbf{partitions} of a number; recall that we briefly talked about these in MATH 25700!
    \end{itemize}
    \item \textbf{Partition} (of $n\in\N$): An ordered tuple satisfying the following constraints. \emph{Denoted by} $\bm{\lambda}$, $\bm{(\lambda_1,\ldots,\lambda_k)}$. \emph{Constraints}
    \begin{enumerate}
        \item $\lambda_i\in\N$ for $i=1,\dots,k$;
        \item $\lambda_1\geq\cdots\geq\lambda_k$;
        \item $\lambda_1+\cdots\lambda_k=n$.
    \end{enumerate}
    \item Example: The partitions of the number "4" are $(4)$, $(3,1)$, $(2,2)$, $(2,1,1)$, and $(1,1,1,1)$.
    \begin{itemize}
        \item This is the same way we've been denoting representations!
    \end{itemize}
    \item $\bm{p(n)}$: The number of possible partitions of $n$.
    \begin{itemize}
        \item Hardy and Ramanujan helped understand the number $p(n)$ of partitions of $n$, but they're still very hard to understand.
    \end{itemize}
    \item One way to understand $p(n)$ is through its encoding in the \textbf{generating function}
    \begin{equation*}
        \sum_{n\geq 1}p(n)x^n = 1+x+2x^2+3x^3+5x^4+\cdots
    \end{equation*}
    \begin{itemize}
        \item We can think of the above generating function as an actual function of $x$ if it converges for small $x$; of it doesn't converge, then we just think of it as a "meaningless" \textbf{formal power series}.
        \item To choose a partition, we need to choose a certain number of 1's, a certain number of 2's, a certain number of 3's, etc. all the way up to $n$.
        \item So let's look at
        \begin{equation*}
            (1+x+x^2+\cdots)(1+x^2+x^4+\cdots)(1+x^3+x^6+\cdots)(1+x^4+x^8+\cdots)\cdots
        \end{equation*}
        \begin{itemize}
            \item Formally, this is
            \begin{equation*}
                \prod_{i=1}^\infty\left( \sum_{j=0}^\infty x^{ij} \right)
            \end{equation*}
            \item This equals the generating function! It tells us that to compute $p(100)x^{100}$, we need only look at certain terms.
        \end{itemize}
        \item Recall that we can write $1+x+x^2+\cdots=1/(1-x)$. Doing similarly for other terms transforms the above product into
        \begin{equation*}
            \frac{1}{1-x}\frac{1}{1-x^2}\frac{1}{1-x^3}\cdots
        \end{equation*}
    \end{itemize}
    \item \textbf{Generating function}: An encoding of an infinite sequence of numbers as the coefficients of a formal power series.
    \item \textbf{Formal power series}: An infinite sum of terms of the form $ax^n$ that is considered independently from any notion of convergence.
    \item The above discussion of $p(n)$ as a generating function is only for our fun; Rudenko is not going to use this in any way.
    \begin{itemize}
        \item It's just a pretty function.
        \item Takeaway: We can write the generating function as something nice and then use it to prove something.
    \end{itemize}
    \item We can visualize these partitions using something called a \textbf{Young diagram}.
    \begin{figure}[h!]
        \centering
        \begin{subfigure}[b]{0.3\linewidth}
            \centering
            \ydiagram{5,4,4,3,2,1,1}
            \caption{$\lambda=(5,4,4,3,2,1,1)$.}
            \label{fig:youngDiagram20a}
        \end{subfigure}
        \begin{subfigure}[b]{0.3\linewidth}
            \centering
            \ydiagram{7,5,4,3,1}
            \caption{$\lambda'=(7,5,4,3,1)$.}
            \label{fig:youngDiagram20b}
        \end{subfigure}
        \caption{Young diagrams for a partition of 20.}
        \label{fig:youngDiagram20}
    \end{figure}
    \begin{itemize}
        \item Suppose we have the following partition of 20: $(5,4,4,3,2,1,1)$.
        \item Then we draw 5 cages for 5 little birds, followed by 4 cages for 4 little birds, etc.
        \item Thus, the $i^\text{th}$ row of boxes has length $\lambda_i$.
        \item The same way you can denote by $\lambda$ the whole \emph{partition}, you can denote by $\lambda$ the whole \emph{diagram}.
        \item This is just a way to visualize partitions.
        \item Recall the three partitions of $S_3$, corresponding to its representations: $(3)$, $(2,1)$, $(1,1,1)$.
        \item Moreover, these diagrams are actually meaningful!
    \end{itemize}
    \item \textbf{Inverse} (of $\lambda$): The partition $(\lambda_1',\dots,\lambda_k')$ defined as follows. \emph{Denoted by} $\bm{\lambda'}$. \emph{Given by}
    \begin{equation*}
        \lambda_i' = |\{\lambda_j\mid\lambda_j>i\}|
    \end{equation*}
    for all $i=1,\dots,k$.
    \begin{itemize}
        \item We can see that $\lambda_1'\geq\cdots\geq\lambda_k'$.
        \item We can also see that the sum will still be $n$.
        \item Moreover, if we do this twice, we'll get back to $\lambda$, i.e., $(\lambda')'=\lambda$.
        \item We can prove $(\lambda')'=\lambda$ combinatorially, too, (that is, without Young diagrams) but that gets pretty complicated.
    \end{itemize}
    \item Example: If $\lambda=(5,4,4,3,2,1,1)$ as above, then $\lambda'=(7,5,4,3,1)$.
    \begin{itemize}
        \item See Figure \ref{fig:youngDiagram20b}.
        \item Moreover, the Young diagrams are related by a flip akin to matrix transposition.
        \item Notice how the definition of inversion \emph{exactly} specifies this flip in the picture: The number of $\lambda_j$'s that have length at least 1 is all the first column of Figure \ref{fig:youngDiagram20a}, the number of length at least 2 is all the second column, etc.
    \end{itemize}
    \item Onto the next question, which is the main miracle.
    \begin{itemize}
        \item Main miracle: There exists a natural (i.e., canonical) bijection between the conjugacy classes and irreducible representations of $S_n$.
        \item We've explored a duality for general finite groups $G$, before, but never a bijection.
        \begin{itemize}
            \item In $S_n$, there \emph{is} this natural bijection.
            \item If you understand why intuitively, you will have started to understand the representation theory of $S_n$.
        \end{itemize}
    \end{itemize}
    \item If we define $\lambda\mapsto n$ (??), then there is some irrep $V_\lambda$ corresponding to $\lambda$. We will look at the \textbf{Specht module} construction of $V_\lambda$.
    \begin{itemize}
        \item Some of the proofs Rudenko will present, he stole from \textcite{bib:Etingof}, and some of the proofs he invented himself.
        \item This is \emph{by far} the best construction, even though it's exceedingly rare in the literature.
    \end{itemize}
    \item The usual construction.
    \begin{itemize}
        \item Take $\C[S_n]$ with coefficients $a_\lambda,b_\lambda$, etc. similar over conjugacy classes and do something with it??
        \item "Just say NO!" to this construction.
    \end{itemize}
    \item Here is the better idea.
    \begin{itemize}
        \item Consider an algebra of polynomials with rational coefficients: $\Q[x_1,\dots,x_n]$.
        \begin{itemize}
            \item We could also do real or complex, but rational is nice.
        \end{itemize}
        \item For symmetric groups, all representations will be integers, etc.??
        \item One thing to emphasize about this algebra: It is a \textbf{graded} algebra.
        \begin{itemize}
            \item If represented by $A$, then it equals $A_0\oplus A_1\oplus A_2\oplus\cdots$ where
            \begin{equation*}
                A_m = \left\{ \sum_{k_1+\cdots+k_n=m}a_{k_1\cdots k_n}x_1^{k_1}\cdots x_n^{k_n} \right\}
            \end{equation*}
            \begin{itemize}
                \item I.e., $A_m$ is the sum of all polynomials with degree equal to $m$.
                \item Example: If we take $1+x_1^2x_2^3+x_1x_2+x_1^{100}+x_1x_2^{99}$, we can then break this polynomial up into polynomials of degree 1, 5, 2, and 100.
            \end{itemize}
            \item We also have $A_{m_1}\cdot A_{m_2}\subset A_{m_1+m_2}$.
            \begin{itemize}
                \item Exmaple: $x_1x_2^2\cdot(x_1+x_2)=x_1^2x_2^2+x_1x_2^3$.
            \end{itemize}
        \end{itemize}
        \item With this algebra in hand, we may let $S_n\acts\Q[x_1,\dots,x_n]$ via
        \begin{equation*}
            \sigma P(x_1,\dots,x_n) = P(x_{\sigma_{-1}(1)},\dots,x_{\sigma^{-1}(n)})
        \end{equation*}
        \begin{itemize}
            \item In other words, $\sigma$ is transposing polynomials.
            \item Example: $(12)(x_1^2+x_2^3+x_3)=x_2^2+x_1^3+x_3$.
        \end{itemize}
        \item Thus, we call as $A_1$ the representation $V_\text{perm}^*$.
        \begin{itemize}
            \item This is because $A_1=\spn(x_1+\cdots+x_n)$, and permuting these is much like permuting the basis of a vector space, as the typical permutation representation does.
            \item It could technically be the isomorphic representation $V_\text{perm}$, but the dual fits better here for reasons??
        \end{itemize}
        \item Then $A_2=S^2V^*$.
        \begin{itemize}
            \item So if $A_1$ had basis $e^1,\dots,e^m$, $A_2$ has basis $e^ie^j$.
            \item Why are we choosing these sets??
        \end{itemize}
        \item Continuing, $A_3=S^3V^*$.
        \item It follows that the representation of the overall thing is
        \begin{equation*}
            \bigoplus_{m\geq 0}(S^mV_\text{perm}^*)
        \end{equation*}
        \begin{itemize}
            \item This is called the \textbf{symmetric algebra}.
        \end{itemize}
    \end{itemize}
    \item \textbf{Graded} (algebra): An algebra for which the underlying additive group is a direct sum of abelian groups $A_i$ such that $A_iA_j\subset A_{i+j}$.
    \item So how do we construct representations?
    \begin{itemize}
        \item For $S_2$, $x_1-x_2$ changes sign when we apply $S_2$.
        \item For $S_3$\dots
        \begin{itemize}
            \item The trivial's polynomial is 1 and \emph{tableaux}.
            \item The standard is $(2,1)$. When we apply $S_3$ to $(x_1-x_2)$, we get
            \begin{equation*}
                \langle (x_1-x_2),(x_2-x_1),(x_1-x_3),(x_3-x_1),(x_2-x_3),(x_3-x_2) \rangle
            \end{equation*}
            \begin{itemize}
                \item If we let $a=x_1-x_2$, $b=x_2-x_3$, then some elements equal $a+b$. This is another way to think about the action.
            \end{itemize}
            \item What about the alternating representation? We have $(x_1-x_2)(x_2-x_3)(x_1-x_3)=\Delta_{123}$, which changes sign when we apply any element with sign $-1$!
        \end{itemize}
        \item For $S_4$\dots
        \begin{itemize}
            \item $(4)$ is 1.
            \item $(3,1)$ is $S_4(x_1-x_2)=\Delta_{12}$.
            \item $(1,1,1,1)$ is $(x_1-x_2)(x_1-x_3)(x_1-x_4)(x_2-x_3)(x_2-x_4)(x_3-x_4)=\Delta_{1234}$.
            \item $(2,1,1)$ is $(x_1-x_2)(x_1-x_3)(x_2-x_3)$.
            \begin{itemize}
                \item We got this polynomial by guessing; the same way $(x_1-x_2)$ worked in multiple cases, maybe this one does too! And it does.
                \item Something to check is that $\Delta_{123}-\Delta_{124}-\Delta_{134}-\Delta_{234}=0$.
            \end{itemize}
            \item $(2,2)$ is $(x_1-x_2)(x_3-x_4)$.
            \begin{itemize}
                \item Something related we can prove is that
                \begin{equation*}
                    (x_1-x_2)(x_3-x_4)-(x_1-x_3)(x_2-x_4)-(x_1-x_4)(x_2-x_3) = 0
                \end{equation*}
                \item This formula appears in \textbf{cross ratios}, which we can discuss in Rudenko's algebraic geometry course next quarter.
            \end{itemize}
            \item For $\lambda=(4,3,1)$, we have $\Delta_{123}\Delta_{45}\Delta_{67}$, and we act by $S_8$ upon this! Explicitly, we have $S_8(x_1-x_2)(x_1-x_3)(x_2-x_3)(x-4-x_5)(x_6-x_7)$.
            \item Takeaway: It all depends on column length!
        \end{itemize}
        \item These polynomials are called \textbf{Vandermonde determinants}; those are the little $\Delta$ things with subscripts. We'll talk about these next times.
    \end{itemize}
    \item We need to prove reducibility and not pairwise isomorphic to make sure that this construction is valid, but that's easy!
\end{itemize}




\end{document}