\documentclass[../notes.tex]{subfiles}

\pagestyle{main}
\renewcommand{\chaptermark}[1]{\markboth{\chaptername\ \thechapter\ (#1)}{}}
\setcounter{chapter}{5}

\begin{document}




\chapter{Abstract Representation Theory}
\section{The Center of the Group Algebra}
\begin{itemize}
    \item \marginnote{10/30:}Plan for this week.
    \begin{itemize}
        \item Today: Briefly discuss a very important concept called the \textbf{center}.
        \item Wednesday: Do algebraic numbers.
        \item Friday: Burnside's theorem.
    \end{itemize}
    \item \textbf{Center} (of a group): The set of all elements of a group $G$ that commute with every other element in $G$. \emph{Denoted by} $\bm{Z(G)}$. \emph{Given by}
    \begin{equation*}
        Z(G) = \{g\in G\mid xgx^{-1}=g\ \forall\ x\in G\}
    \end{equation*}
    \begin{itemize}
        \item Note: $Z(G)$ is a subgroup of $G$.
    \end{itemize}
    \item The center is one of the most important concepts in all of representation theory.
    \begin{itemize}
        \item Example: Let $A$ be an abelian group, such as $Z(G)$. Then all its irreps are 1D.
        \begin{itemize}
            \item See Section 1.3 of \textcite{bib:FultonHarris} for an explanation.
        \end{itemize}
        \item Normally, the center of a group is too small to be interesting.
        \begin{itemize}
            \item However, $Z(\C[G])$ \emph{is} large enough to be interesting.
        \end{itemize}
    \end{itemize}
    \item \textbf{Center} (of an algebra): The set of all elements of an algebra $A$ that commute with every other element in $A$. \emph{Denoted by} $\bm{Z(A)}$. \emph{Given by}
    \begin{equation*}
        Z(A) = \{a\in A\mid xa=ax\ \forall\ x\in A\}
    \end{equation*}
    \item Proposition: If $A$ is an algebra over $\C$, $M$ is an irreducible left $A$-module, and $\rho:A\to\End(M)$ is a corresponding representation, then $x\in Z(A)$ implies that $\rho(x)=\lambda I$, i.e., $\rho(x)$ is a \emph{scalar matrix}.
    \begin{proof}
        Let $x\in Z(A)$ be arbitrary. Then for all $a\in A$, we know that $\rho(x)\rho(a)=\rho(a)\rho(x)$. Thus, $\rho(x)$ is a morphism of $A$-modules. Consequently, since $M$ is irreducible (also known as \emph{simple}), Schur's Lemma for associative algebras implies that $\Hom_A(M,M)$ is a division algebra over $\C$. But since $\C$ is the only division algebra over $\C$, we have that $\Hom_A(M,M)\cong\C$. From here, it readily follows that $\rho(x)$ is equal to some $\lambda I$.
    \end{proof}
    \item Consequence: If $M$ is reducible, we can reduce it into component scalar representations.
    \item Consequence: If $G$ is an abelian group, then every irrep $V$ is 1-dimensional.
    \begin{itemize}
        \item Additionally, $\C[G]$ is commutative and hence $\C[G]=Z(\C[G])$.
        \item Then if $V$ is an arbitrary representation, $V$ is equal to the direct sum of one dimensional irreducible representations for all $g$. Hence, $\rho_V(g)=\lambda I$. Could the $\lambda$'s not be different for the various irreps??
    \end{itemize}
    \item We now try to compute $Z(\C[G])$.
    \begin{itemize}
        \item Facts:
        \begin{align*}
            Z(A_1\oplus A_2) &= Z(A_1)\oplus Z(A_2)&
            Z(M_n(\C)) &= \spn(I) \cong \C
        \end{align*}
        \item These facts coupled with the fact that $G$ is a finite group (hence $\C[G]\cong M_{n_1}(\C)\oplus\cdots\oplus M_{n_k}(\C)$ where $k$ is the number of conjugacy classes in $G$ by the example from last Wednesday's class) yield
        \begin{align*}
            Z(\C[G]) &\cong Z(M_{n_1}(\C)\oplus\cdots\oplus M_{n_k}(\C))\\
            &\cong \underbrace{\C\oplus\cdots\oplus\C}_{k\text{ times}}\\
            &= \C^k
        \end{align*}
    \end{itemize}
    \item Let $C_1,\dots,C_k$ be conjugacy classes in $G$. Then we may define
    \begin{equation*}
        e_i = \sum_{g\in C_i}g
    \end{equation*}
    for each $i=1,\dots,k$.
    \begin{itemize}
        \item Example: In $S_3$ the three $e_i$'s are $\{e,(12)+(13)+(23),(123)+(132)\}$.
    \end{itemize}
    \item Claim: $Z(G)=\inp{e_1,\dots,e_k}$, that is, the $e_i$ commute with every element of $G$ expressed as $1g\in\C[G]$.
    \begin{proof}
        We will use a bidirectional inclusion proof.\par
        \underline{$\inp{e_1,\dots,e_k}\subset Z(G)$}: Let $e_i$ and $x\in G$ be arbitrary. Then
        \begin{align*}
            % xe_i &= \left( \sum_{g\in G}a_gg \right)e_i\\
            % &= \left( \sum_{g\in G}a_gg \right)\left( \sum_{h\in C_i}h \right)\\
            % &= \sum_{g\in G}\left[ a_gg\left( \sum_{h\in C_i}h \right) \right]\\
            % &= \sum_{g\in G}\left[ a_g\left( \sum_{h\in C_i}gh \right) \right]\\
            % &= \sum_{g\in G}\left[ a_g\left( \sum_{h\in C_i}hg \right) \right]\\
            % &= \sum_{g\in G}\left[ a_g\left( \sum_{h\in C_i}h \right)g \right]\\
            % &= \left( \sum_{h\in C_i}h \right)\left( \sum_{g\in G}a_gg \right)\\
            % &= e_ix
            xe_ix^{-1} &= \sum_{g\in C_i}xgx^{-1} = \sum_{h\in C_i}h = e_i\\
            xe_i &= e_ix
        \end{align*}
        This naturally extends to any sums and scalar multiples of the $e_i$'s.\par
        \underline{$Z(G)\subset\inp{e_1,\dots,e_k}$}: Let $a\in Z(G)$ be arbitrary. As an element of $\C[G]$, we know that $a=\sum a_gg$ for some $a_g\in\C$. Additionally, since $a\in Z(G)$, we have that $xax^{-1}=a$ for all $x\in G$ (that is, $1x\in A$). Combining these last two results, we have that
        \begin{equation*}
            \sum_{g\in G}a_{x^{-1}gx}g = \sum_{g\in G}a_gxgx^{-1}
            = xax^{-1}
            = a
            = \sum_{g\in G}a_gg
        \end{equation*}
        Comparing like terms in the above equality, we can learn that for all $x\in G$, we have $a_{x^{-1}gx}=a_g$. In other words, all of the $a_g$'s for $g$'s in the same conjugacy class are equal. Therefore, $a$ is of the form $a=\sum_{i=1}^ka_{g_i}e_i$ for $g_i\in C_i$.
    \end{proof}
    \item Thus we get $a_ee+a_{(12)}(12)+a_{(13)}(13)+\cdots$??
    \item Computing products of the $e_i$: What if we want to compute $[(12)+(13)+(23)]^2$, for example? We have to multiply \emph{noncommutatively}, so HS formulas are out, but we can still do all nine multiplications and sum them:
    \begin{equation*}
        [(12)+(13)+(23)]^2 = 3e+3[(123)+(132)]
    \end{equation*}
    \item We now tie this claim back into our discussion of $Z(\C[G])$.
    % \item Claim: $Z(\C[G])=\inp{e_1\dots,e_k}$.
    % \begin{proof}
    %     \underline{$\inp{e_1,\dots,e_k}\subset Z(\C[G])$}: Let $e_i$ and $x=\sum_{g\in G}a_gg\in\C[G]$ be arbitrary. Then
    %     \begin{equation*}
    %         xe_i = \sum_{g\in G}a_gge_i
    %         = \sum_{g\in G}a_ge_ig
    %         = e_i\sum_{g\in G}a_gg
    %         = e_ix
    %     \end{equation*}
    %     where the second equality holds because of the claim above.\par
    %     \underline{$Z(\C[G])\subset\inp{e_1,\dots,e_k}$}: Let $a\in Z(G)$ be arbitrary. As an element of $\C[G]$, we know that $a=\sum a_gg$ for some $a_g\in\C$. Additionally, since $a\in Z(\C[G])$, we have that $xa=ax$ for all $x=\sum_{g\in G}a_gg\in\C[G]$...
    % \end{proof}
    % \begin{itemize}
    %     \item In particular, $Z(G)=\inp{e_1,\dots,e_k}$ implies that $Z(\C[G])=\inp{e_1\dots,e_k}$
    % \end{itemize}
    % \item Consider $\C[G]$.
    \begin{itemize}
        \item $Z(\C[G])$ has basis $e_1,\dots,e_k$\footnote{How did we get from the previous claim to here??}.
        \item Recall that $Z(\C[G])=\C\oplus\cdots\oplus\C$, with characters $\chi_1,\dots,\chi_k$.
        \item Then $f_{\chi_i}=(0,\dots,0,1,0,\dots,0)$, where the 1 lies in the $i^\text{th}$ slot.
        \item Then we get $f_{\chi_1},\dots,f_{\chi_k}$ as a basis.
        \item It follows that $f_{\chi_i}^2=f_{\chi_i}$ and $f_{\chi_i}f_{\chi_j}=0$ for $i\neq j$; this is exactly what it means for a space to be $\C\oplus\cdots\oplus\C$.
        \item Both of these spaces (center elements and class functions) have these two interconnected bases, so the spaces are quite similar!
    \end{itemize}
    \item The center of a group algebra $Z(\C[G])$ can be identified "$=$" with the space of class functions $\C_\text{cl}(G)$ via
    \begin{equation*}
        \sum\varphi(g)g \mapsto [g\to\varphi(g)]
    \end{equation*}
    where $\varphi(xgx^{-1})=\varphi(g)$.
    \begin{itemize}
        \item This isomorphism is an isomorphism of vector spaces, \emph{not} an isomorphism of algebras!
        \item However, it still has cool properties.
        \begin{itemize}
            \item For instance, consider the $\delta_{C_i}$: The functions sending $g\in C_i$ to 1 and $g\notin C_i$ to 0.
            \item The isomorphism identifies $e_i\mapsto\delta_{C_i}$.
        \end{itemize}
        \item Do we get irreducible characters (our other basis of class functions) when we sum the $\varphi(g)g$'s?
        \begin{itemize}
            \item We do! What is this??
        \end{itemize}
        \item Let's consider another basis $\chi$ of irreducibles. The basis is $f_\chi=\frac{d_\chi}{|G|}\sum_{g\in G}\chi(g^{-1})g$, and we send it to $\chi_{V^*}$.
        \item Claim:
        \begin{equation*}
            f_{\chi_i}f_{\chi_j} =
            \begin{cases}
                f_{\chi_i} & \chi_i=\chi_j\\
                0 & \chi_i\neq\chi_j
            \end{cases}
        \end{equation*}
        \begin{itemize}
            \item Things that multiply like this are called the \textbf{central idempotent}.
        \end{itemize}
        \item Thus, general multiplication works as follows.
        \begin{equation*}
            (a_1f_{\chi_1}+\cdots+a_nf_{\chi_n})(b_1f_{\chi_1}+\cdots+e_nf_{\chi_n}) = a_1b_1f_{\chi_1}+\cdots+a_nb_nf_{\chi_n}
        \end{equation*}
        \item So if we want to send $a\in Z(G)$ to $\bigoplus^k\C$, we map
        \begin{equation*}
            a=a_1f_{\chi_1}+\cdots+a_kf_{\chi_k} \mapsto (a_1,\dots,a_k)
        \end{equation*}
        \item The proof of this claim is really simple because we've already done the computation with the projector on the irrep $V_x$.
        \begin{itemize}
            \item So if you want to see $\rho(f_\chi)$, see what it does to the identity: It does $\rho(f_\chi)e=f\chi e=f_\chi$. $\rho$ is regular.
        \end{itemize}
    \end{itemize}
    \item \textbf{Central idempotent}: An element such that $a^2=a$ and $ax=xa$ for all $x\in A$.
    \item Two approaches to the same thing: Class functions and the center approach.
    \begin{itemize}
        \item The great thing about the center: You can understand what it looks like because it is well-defined as a commutative algebra.
        \item If something is isomorphic to $\C\oplus\cdots\oplus\C$ as an algebra, then there is another space and basis in which your multiplication looks incredibly simple.
    \end{itemize}
    \item We might get to \textbf{Hopf algebras} at the end of the course (very interesting).
    \begin{itemize}
        \item Let $\C[G]$ be an associative algebra.
        \item Let $\C[G]^*$ be the functions on the group.
        \item Then $A\otimes A\to A$ sends $a_1\otimes a_2\mapsto a_1a_2$.
        \item When we dualize to get $A^*\otimes A^*\to A^*$, everything gets reversed, so we actually get a \textbf{comultiplication} $A\to A\otimes A$ given by $g\mapsto g\otimes g$. These two multiplications together are called a \textbf{Hopf algebra}.
        \item Knowing that there's something that we can define and understand might help us untangle the knot of all the spaces.
        \item This is pretty heavy math, though, so we won't go too deep into it if we get at all.
    \end{itemize}
    \item Today was the last associative algebra class.
    \item Going forward: Integral elements, algebraic integers, dimension of the representation divides the order or the group, Burnside's theorem.
    \item Midterm is heavily computational: Tensor products, character tables, etc. A few simple questions about things.
    \begin{itemize}
        \item Comparably less associative algebra stuff (maybe just 1 exercise).
    \end{itemize}
\end{itemize}




\end{document}